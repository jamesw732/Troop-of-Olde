\documentclass{article}
\usepackage{graphicx}
\usepackage{enumerate}
\usepackage{hyperref}
\usepackage{xparse}
\usepackage{amsmath}

\NewDocumentCommand{\codeword}{v}{%
\texttt{{#1}}%
}

\title{Troop of Olde Documentation}
\author{James Westbrook}

\begin{document}
\maketitle

\tableofcontents

\pagebreak

\section{Development Plan}
I expect pre-release development to basically take place in three stages.
I don't expect to confine
myself to these stages, but I think they will help structure my development. Currently
in the first stage, maybe like 50\% of the way through it.

\subsection{Stage 1}
First is the prototyping stage, where I simply try to implement the relevant features
of the game. Building combat mechanics, physics, networking, all that sort of stuff.
The key is functionality, not aesthetics. Animations and models will play a role in this stage,
but only in basic form. All art will be placeholder or lacking. UI is solely meant to provide an
interface between the user and functionality, and does not need to be pretty.
Checklist for this phase:
\begin{itemize}
    \item Networking
    \item Physics
    \item Auto attacking
    \item Sample Powers (one of each type)
    \item Items
    \item Robust NPC/player deaths
    \item Looting
    \item Learning powers from scrolls (and moving them on action bar)
    \item Basic stats functionality
    \item Basic skills functionality
    \item UI that interfaces all of the above
    \item Character model prototype and movement/attack animations
    \item Test connections on more than one device
    \item NPC AI
    \item Research necessary next steps such as art styles, blender,
    music composition, procedural generation
\end{itemize}
At the end of this phase, this will not be a game, it will be more like a blueprint for a game.
Should also consider closing the source code after declaring this phase to be done.

\subsection{Stage 2}
Second is the content generation stage. The basic idea here is to get a minimal working
example for the game. Once the basic prototypes and mechanics are nailed down,
it's time to start extending them into actual content used to build the world.
A necessary transition step here will be reconsidering the database. I don't think
it will be good to just use json files for everything in the final product, so the first
step before creating content will be to improve this. Since the plan is to make use of procedural
generation to make content, there will need to be a lot of time spent making this work during
this stage. Modeling, animations, and art will be a big part of this stage, but it will be
important to balance creating art with generating content. UI will probably not be a big
part of this phase, since the UI is mostly attached to the functionality.
Checklist for this phase:
\begin{itemize}
    \item Reconsider/improve database
    \item Create or generate world
    \item Create or generate items
    \item Create or generate NPCs
    \item Create powers
    \item Make physics more robust to the world, and consider improving the feel of movement.
    \item Create launch screen for client and server
    \item Implement save system
    \item Add more models and character animations
    \item Add animations for powers (mostly just particles)
    \item Explore art style and maybe experiment with procedural art
    \item Add real art to the UI
    \item Start composing music and making sound effects
    \item Improve network security/reliability
\end{itemize}
At the end of this stage, there should be a playable but incomplete game.
I should place a high standard on the quality of art entering the game. At the end of this stage,
it should look like a marketable product. It doesn't need to be a finished product, but I should
be able to release a Steam page and be able to find playtesters.

\subsection{Stage 3}
Third is the release stage. The goal here is to enter alpha and push out a finished product.
This is far-out still, so I'm guaranteed to miss a lot here, but I'm expecting some things to
be certain.
Checklist for this phase:
\begin{itemize}
    \item Create Steam page and market the game
    \item Improve quality of assets as needed
    \item Improve game balance
    \item Consider outsourcing cloud bandwidth to allow players to host servers remotely
    \item Finish composing music
\end{itemize}

\section{Code Infrastructure}
This section describes the key development choices made in the source code
for this game.  This is organized by ``code layers'', which are essentially just
mental folders I use to try to organize the code.

\subsection{Project Root}
These are the \codeword{main.py}, \codeword{server.py}, \codeword{main_multiplayer.py}
files which pretty much just import everything necessary initialize the game. 
There is not currently any build process. This is subject to change.

The recommended way to launch the client is by simply running \codeword{main.py}, which
launches a child \codeword{server.py} process and connects to it automatically.
If you'd like to connect to the same local server on multiple clients, you can then run
\codeword{main_multiplayer.py} and press "c" to connect.

Alternatively, you may run \codeword{server.py} as its own process, and then connect
to it from \codeword{main_multiplayer.py}.

Of course, this launch process is subject to change once this becomes more of a real game.

\subsection{External API}
The external API is the gateway between ``external'' inputs and the internal game logic.
Namely, this mainly includes the whole UI module and \codeword{controllers.py}, but I
also include the networking module.

The role of the UI module is to display necessary parts of the game state, 
handle relevant inputs from the player, and send networking requests. Networking responses
are capable of updating the state of the UI, for example when the player's health is updated,
the health bar decreases.

The role of \codeword{controllers.py} is to perform operations which affect the state of
Characters. Controllers are attached to single Characters. They are not sentient, but they
do respond to inputs.\\
PlayerController handles (some) player input, updating the local game state and sending
network requests. Networking responses is able to modify the state of PlayerController as
well, for example the controller is responsible for the interpolation that corrects
client-side prediction for player movement.\\
NPCController handles networking responses for client-side NPCs.\\
MobController is the server-side Character controller. It handles server-side operations such as
combat and physics.

The role of the networking module is, really, to keep the client and server on the same page
about the game state. But the way it does this is by interfacing between the functions of the
external API. So it's ``sort of'' part of the external API, or more like the glue that keeps
it together. See below for an in-depth description of the networking module.

Eventually, NPC logic will be built into an external server-side only class, and not be part
of MobController. This won't be part of the external API layer, since the role of an NPC logic
class is to emulate ``player'' inputs.

I am not extremely happy with the structure of this layer. I think the input handling should be
moved into a separate class, which would call Controller methods and send network requests.
This class would be solely responsible for sending network responses, so there would be none
in the external API. This paradigm would allow for a cleaner divide between the code layers:
The input handler would be the highest logical layer under the root, then the networking module,
then the external API, and finally the internal API.

\subsubsection{Networking}
This game runs on a client-server framework. Currently, it is only tested to
support one device launching multiple clients.
All networking code is built on Ursina's \codeword{networking} submodule, which
is built on TCP and based on remote procedure calls (RPCs). It is located in
the \codeword{networking} submodule of this repository. The organization is
under consideration, but below is a brief description of the files in this
directory:
\begin{itemize}
    \item \codeword{connect.py} is a small file that handles the work required
        when establishing a new connection to the server. It essentially just
        defines the \codeword{on_connect} function and an \codeword{input}
        function for connecting. Eventually, this input function will be
        extracted into an external login sequence that uses actual GUI,
        but this is low priority right now.
    \item \codeword{disconnect.py}, similarly, handles the work required when
        leaving the server. Disconnection logic is currently buggy, but this
        is a low priority at the moment.
    \item \codeword{network.py} defines the \codeword{Network} class, which is
        mostly just a centralized location for crucial network information
        such as the \codeword{Peer} instance and dicts mapping ids and connections
        to objects in memory. The \codeword{Network} object is added as a
        variable of the \codeword{GameState} object, so that RPCs can be accessed
        without having to import the corresponding files. This provides a great
        deal of separation between the internal game logic and the networking
        code at the cost of a global variable.
    \item \codeword{register.py} is a simple procedural file that registers all
        custom data types to be sent across the network. For the most part, these
        data types are \codeword{State}s, which will be explained later, but these
        are essentially just large groups of arguments attached together in a single
        object.
    \item \codeword{world_requests.py} is a file containing all in-game networking
        requests sent by clients. All functions in this file are remote procedure
        calls that are {\it called} by clients, and {\it executed} on the server.
        Relies on the game code and on \codeword{world_responses.py}.
    \item \codeword{world_responses.py} is a file containing all in-game networking
        responses sent by the server. All functions in this file are remote procedure
        item calls that are called by the server, executed on a client. Relies on the
        item game code.
\end{itemize}
See the top-level files \codeword{server.py}, \codeword{main_multiplayer.py}, and
\codeword{main.py} to see how connections are made.

\subsection{Internal API}
The internal API governs the internal game mechanics. This is the largest layer,
many of the files deserve their own explanations.

\subsubsection{States}
States are essentially simplified versions of objects that are sendable over
the network. The typical usage is like:\\
Object $\rightarrow$ State $\rightarrow$ RPC call $\rightarrow$ State
$\rightarrow$ Object,\\
where the last arrow represents either an overwrite of a portion of the object's state,
or applying the state as a difference.

States are defined in \codeword{state_defs.py}. They are currently based on dicts,
but I'm realizing as I type this that this is unnecessary, and could be based on
lists. State definitions are basically mappings of variables to types, and as long as
the state is filled and the ordering is guaranteed and consistent between the
client and server, we can represent a State with just a list. Would need to convert
dicts to lists of tuples.
\subsubsection{Physics}
Implementing physics has proved to be challenging, but the current implementation seems
good.

There are currently three (well, 4, but I'll get to the fourth later) movement
factors. Player keyboard (WASD) movement, jumping, and gravity. Velocity
components are multiplied by the physics timestep (1/20th of a second) and then
added together, along with the displacement factors. After we obtain a single
velocity vector, we send it to the collision logic.

Collision rules are different when the character is grounded or not.
When not grounded, the character essentially just looks in the direction it's
falling, and if it would clip into the floor, it adjusts its movement to not clip
into the floor and sets the \codeword{grounded} boolean to True.
When grounded, it's a lot more complicated. We look ahead of where our initial
displacement would take us, and if there is anything in the way, first check
whether it's a wall or not (normalized $x$ component less than 0.2). If it's a wall,
we essentially move along the wall without changing our $y$ height.
The formula for this is obtained by a plane intersection, the intersection of the
plane $ax + by + cz = 0$ and $y = 0$. This yields the parameterization
\[
\begin{pmatrix}
    x\\
    y\\
    z 
\end{pmatrix} =
\begin{pmatrix}
    b\\0\\-a
\end{pmatrix}s,\]
and rescaling this and multiplying by the dot product gives us the displacement
we want.

When the colliding surface is a feasible slope to walk up, we instead keep
the $(x, z)$ direction from our original movement, and deduce the correct
$y$ from this and the plane normal. This is the same as finding the
intersection of the planes $ax + by + cz = 0$ and the plane defined by
$(0, 0, 0), (0, 1, 0), (\alpha, 0, \beta)$ (we look straight up because gravity goes
straight down), where $(\alpha, \beta)$ is the $(x, z)$ direction of our
original displacement. Then our line is defined by
\[
\begin{pmatrix}
    x\\y\\z
\end{pmatrix} = 
\begin{pmatrix}
    \alpha b\\-\beta c - \alpha a\\\beta b
\end{pmatrix}s.\]
Rescaling to match our original displacement gives us our final result.
\subsubsection{Characters}
Characters have different classes depending on whether they were created on the client
or the server.
\subsubsection{Client-Side Prediction}
To ensure smooth play while enforcing server-authoritative character movement,
we simulate the game physics client-side. Unfortunately, the simulation
is nondeterministic, so we need to adjust for the differences in the state.
The process for client-side prediction is as follows:\\
On every physics tick, we send the movement inputs from this tick to
the server, and then simulate the physics on the client. We store the
current position/rotation, and the ``targets'', and linearly interpolate
between them over the duration of the physics tick. This introduces a
physics tick worth of lag, but it's not noticeable and inconsequential
to the gameplay. This constitutes the ``prediction'' side of this story.

To accommodate the server updates, when we send the movement inputs, we
also include a sequence number that increments by one each physics
tick. We map the sequence number to the current target state, and
when the server sends back the state, it also sends back the current
highest sequence number it has. Of course, when the state arrives back
to the client, it's in the past, so we look back in time to what our
target state was at the retransmitted sequence number. We compute the
difference, and add it to the current target. Crucially, we add the
displacement to the \codeword{Character}'s \codeword{displacement_components}
to be processed for collision logic. Without applying the collision logic, it's easy to clip through the world and have an irrecoverable state.

\subsection{Design Patterns}
To deal with circular dependencies in the networking code, I have chosen to attach top-level objects to
the global GameState object. This allows referencing and using these objects without importing the
corresponding modules. This is a dangerous choice, but I have found it to be a necessary evil (or maybe
just too difficult for me to avoid). The greatest cost is that it blurs the line between the different
layers. Actually, a big part of why I've focused on documenting the layers is to give the code design
more structure, because this choice could, if abused, create some eldritch spaghetti code.
The choice to use layers is not out of immediate necessity, but rather to prevent bad habits before
they arise. 


\section{Ursina Usage}
\subsection{Common Bugs}
\begin{itemize}
    \item ``Remote attempted to call a RPC that does not exist (maybe a name typo in the source code?)''\\
        This essentially just means that an unregistered RPC was called. Unfortunately it does not give
        the name of the procedure. Usually this is fixed by importing the relevant file that contains the
        RPC definition, fixing a typo, or implementing a non-existing RPC.
\end{itemize}
\end{document}
