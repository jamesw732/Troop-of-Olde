\documentclass{article}
\usepackage{graphicx}
\usepackage{enumerate}
\usepackage{hyperref}
\usepackage{xparse}
\usepackage{amsmath}
\usepackage[strings]{underscore}

\NewDocumentCommand{\codeword}{v}{%
\texttt{{#1}}%
}

\title{Troop of Olde Documentation}
\author{James Westbrook}

\begin{document}
\maketitle

\tableofcontents

\pagebreak

\section{Development Plan}
I expect pre-release development to basically take place in three stages.
I don't expect to confine
myself to these stages, but I think they will help structure my development. Currently
in the first stage, maybe like 50\% of the way through it.

\subsection{Stage 1}
First is the prototyping stage, where I simply try to implement the relevant features
of the game. Building combat mechanics, physics, networking, all that sort of stuff.
The key is functionality, not aesthetics. Animations and models will play a role in this stage,
but only in basic form. All art will be placeholder or lacking. UI is solely meant to provide an
interface between the user and functionality, and does not need to be pretty.
Checklist for this phase:
\begin{itemize}
    \item Networking
    \item Physics
    \item Auto attacking
    \item Sample Powers (one of each type)
    \item Items
    \item Robust NPC/player deaths
    \item Looting
    \item Learning powers from scrolls (and moving them on action bar)
    \item Basic stats functionality
    \item Basic skills functionality
    \item UI that interfaces all of the above
    \item Character model prototype and movement/attack animations
    \item Test connections on more than one device
    \item NPC AI
    \item Research necessary next steps such as art styles, blender,
    music composition, procedural generation
\end{itemize}
At the end of this phase, this will not be a game, it will be more like a blueprint for a game.
Should also consider closing the source code after declaring this phase to be done.

\subsection{Stage 2}
Second is the content generation stage. The basic idea here is to get a minimal working
example for the game. Once the basic prototypes and mechanics are nailed down,
it's time to start extending them into actual content used to build the world.
A necessary transition step here will be reconsidering the database. I don't think
it will be good to just use json files for everything in the final product, so the first
step before creating content will be to improve this. Since the plan is to make use of procedural
generation to make content, there will need to be a lot of time spent making this work during
this stage. Modeling, animations, and art will be a big part of this stage, but it will be
important to balance creating art with generating content. UI will probably not be a big
part of this phase, since the UI is mostly attached to the functionality.
Checklist for this phase:
\begin{itemize}
    \item Reconsider/improve database
    \item Create or generate world
    \item Create or generate items
    \item Create or generate NPCs
    \item Create powers
    \item Make physics more robust to the world, and consider improving the feel of movement.
    \item Create launch screen for client and server
    \item Implement save system
    \item Add more models and character animations
    \item Add animations for powers (mostly just particles)
    \item Explore art style and maybe experiment with procedural art
    \item Add real art to the UI
    \item Start composing music and making sound effects
    \item Improve network security/reliability
\end{itemize}
At the end of this stage, there should be a playable but incomplete game.
I should place a high standard on the quality of art entering the game. At the end of this stage,
it should look like a marketable product. It doesn't need to be a finished product, but I should
be able to release a Steam page and be able to find playtesters.

\subsection{Stage 3}
Third is the release stage. At this point, the goal is to market the game, improve balance,
and make everything look and feel good. Entering this stage, the game should be ``finished'' in
terms of content and mechanics, but ``unpolished''. Of course, I don't expect to never make
content for the game after its release, but this encapsulates the final steps before release.
Checklist for this phase:
\begin{itemize}
    \item Create Steam page and market the game
    \item Improve quality of assets as needed
    \item Improve game balance
    \item Consider outsourcing cloud bandwidth to allow players to host servers remotely
    \item Finish composing music
\end{itemize}

\section{Code Infrastructure}
This section describes the big-picture structure of the code.

When possible, I try to adhere to a downward dependency structure. Basically, a file
should only store objects lower than it in the dependency tree (ie, objects of a type defined
in a file you could import from). For example, \codeword{Character} should not store its
\codeword{controller}. While it can be quicker to hack together a solution that violates downward
dependence, it typically makes a big mess of tightly coupled spaghetti code. Ownership becomes unclear,
and overall structure becomes complicated (because there is circular dependence all around). Adhering
to the downward dependence gives the codebase a much richer structure, making it more complex but less
complicated (counterintuitively).

The most damning symptom of breaking downward dependence is that it makes cleanup a lot messier.
On the other hand, this is not true when higher level objects are passed as function arguments but
not stored. In this case, we don't have an extra reference to clean up when the higher level
object is destroyed. While it still obfuscates the project structure, I am much more lenient
with this pattern because it is a lot more convenient. It is certainly possible to abuse this
paradigm, so I pretty much only do this for the lower-level game logic.  For example, a lot of
\codeword{Item} functions rely on \codeword{Character} objects, but \codeword{Character} relies
on \codeword{Item} objects because the \codeword{Character} stores inventory and equipment, which
contain \codeword{Items}. This type
of tight coupling is pretty forgivable, especially in Python. The cost of avoiding it seems to outweigh
the benefits to me, so I choose to utilize it. So the general rule is that if an object stores another
object, it should (be able to) import the lower level object's file. If it can't due to circular import,
that means the code is too tightly coupled and needs refactoring.

The other guiding principle is to keep differing behavior between the client and server as disjoint as possible.
The client shouldn't even have access to the server code. Early on in this project's life, I used a lot of checks
to see whether the network was being hosted or not. These should be avoided. If there's a distinction between
the client and server behavior, it should be captured by subclasses that aren't accessible to one another.
At the same time, behavior that is truly identical between the client and server should be reused whenever
possible.

One benefit of the project having a clear dependency stack is that it gives a clear order to present the code in.
The rest of the section describes the entire dependency stack from the bottom up, with brief descriptions of the roles
of each file.
\subsection{Shared API}
\subsubsection{base.py}
This file defines basic constants, template objects, and functions for any file to import.
This is the lowest level file in the project.
\subsubsection{states.py}
States are essentially lists of arguments that make building large objects
and sending RPCs easier.

The base State class defines the blueprint for States, but should never be initialized.
Instead, subclasses give the definition for what the State will contain, and
allow for flexibility in how they are constructed.

Every State is able to read from a \codeword{src} object, which is either a Character
or a dict (these possibilities may grow). They read based on the type, but are always
written as a list. The order of the list is defined by its \codeword{statedef}.

States can also be applied to Characters by overwriting the attributes, or by adding them
(for example, Stats).
\subsubsection{character.py}
This file defines the base Character class, which represents an in-game character.
This class is not intended to be initialized, instead it is just the intersection of
client characters and server characters. Client and server-specific behavior is defined
in ClientCharacter and ServerCharacter classes, but to handle the added complexity of
going through the network, these classes should only be instantiated by the corresponding
World classes, outside of unit testing. The processes for this are detailed in the client
and server character.py sections.

This class does not take any arguments for initialization, instead it sets default values for
all attributes, and defines shared functionality between ClientCharacter and ServerCharacter.

While this class primarily contains small helper functions, it also defines behavior for
moving items between inventory and equipment.
\subsubsection{power.py}
This file contains the definition of the Power class. This class represents Character
abilities/spells, which are ways to interact with other Characters in combat.
Powers spawn an Effect on the server side (eventually client side too), and are
managed by controllers.py - this file essentially just defines the API for the
controllers to use.
\subsubsection{physics.py}
This file contains the physics API, used by \codeword{PlayerController} and
\codeword{MobController}. This file does not drive physics, it just defines the possible
behaviors involved with physics.

Physics is probably the most technically challenging module to understand in this
repository, so I'll devote a large explanation here.

There are currently three (well, 4, but I'll get to the fourth later) movement
factors. Player keyboard (WASD) movement, jumping, and gravity. Velocity
components are multiplied by the physics timestep (1/20th of a second) and then
added together, along with the displacement factors. After we obtain a single
displacement vector, we send it to the collision logic.

We first send a ray in the direction of the displacement vector.
If there's a hit, we infer whether it's a wall or not based on its y normal -
less than 0.2 means it's pretty vertical. If it is a wall, we project the
displacement vector onto the horizontal line along the wall, and treat the character
as grounded.
The formula for this is obtained by a plane intersection, the intersection of the
plane $ax + by + cz = 0$ and $y = 0$. This yields the parameterization
\[
\begin{pmatrix}
    x\\
    y\\
    z 
\end{pmatrix} =
\begin{pmatrix}
    b\\0\\-a
\end{pmatrix}s,\]
and rescaling this and multiplying by the dot product gives us the displacement
we want.

If the colliding surface is a feasible slope to walk up, then we change
our handling depending on whether we are grounded or not. If we are grounded,
then we simply overwrite the displacement with the displacement obtained
by subtracting the character position from the collision point, and
add a small ($10^{-3}$) amount of y displacement to make sure we don't
clip through.

If we are already grounded, then we walk up the slope. We do this by keeping
the $(x, z)$ direction from our original movement, and deduce the correct
$y$ from this and the plane normal. This is the same as finding the
intersection of the planes $ax + by + cz = 0$ and the plane defined by
$(0, 0, 0), (0, 1, 0), (\alpha, 0, \beta)$ (we look straight up because gravity goes
straight down), where $(\alpha, \beta)$ is the $(x, z)$ direction of our
original displacement. Then our line is defined by
\[
\begin{pmatrix}
    x\\y\\z
\end{pmatrix} = 
\begin{pmatrix}
    \alpha b\\-\beta c - \alpha a\\\beta b
\end{pmatrix}s.\]
Rescaling to match our original displacement gives us our final result.
\subsubsection{combat.py}
This file defines the combat API, used by \codeword{MobController}. This file does not
drive combat, it just defines the possible behaviors involved with auto attacking.

\subsubsection{network.py}
This file defines the Network class, which provides an RPCPeer attribute (for interfacing
RPCs), maps between uuids/connections, server connection and player's uuid (unused by
the server), some functions to help broadcast RPCs, and registers all State types to
be used over the network. The network is updated 20 times per second.

\subsection{Client Subpackage}
\subsubsection{character.py}
This class defines the client-specific Character behavior. The process for creating a
ClientCharacter with the server is roughly:
\begin{enumerate}
    \item Call \codeword{World.load_player_data} to pull data from players.json into something
        sendable over the network.
    \item Call \codeword{Network.peer.request_enter_world} with the login data.
    \item Server will call \codeword{World.make_char_init_dict} to get ServerCharacter
        arguments and call \codeword{World.make_char} to initialize and store a ServerCharacter.
    \item Server will extract necessary spawn data from the new character (different
        depending on whether sending to the source client or a different client), and call
        \codeword{Network.peer.spawn_pc} (\codeword{Network.peer.spawn_npc}) for other
        clients.
    \item The client will then call \codeword{World.make_pc_init_dict} to get the
        ClientCharacter arguments, and then call \codeword{World.make_pc} to initialize
        and store the ClientCharacter.
\end{enumerate}
The differences between player characters, player characters for other clients, and NPCs
are not represented at all in \codeword{ClientCharacter}, but this is subject to change.
Currently, the differences are only present in the initialization data and the controllers.

To handle animation, the character model is loaded in as a Panda3D Actor rather than an Entity,
which can be set by specifying a \codeword{model_name}.
\subsection{animator.py}
The character model is animated by the \codeword{Anim} class, which is owned by the client-side
controllers. This class implements a custom animation blending system on top of Panda3D's
\codeword{Actor.setControlEffect}. Animations aren't really stopped, instead they are ``faded
out'', which simply sets its blending coefficient to zero over a given amount of time.
Furthermore, to start an animation, you must ``fade in'' that animation in addition to calling
\codeword{Actor.play} or \codeword{Actor.loop}.

For multi-part animations, create subparts using \codeword{Actor.makeSubpart} and use the
part name as an argument to the \codeword{partName} parameter of \codeword{Actor.play} or
\codeword{Actor.loop}. It's important to be very careful about which animations affect
which part, otherwise the blending will be incorrect. See the attack animation code for
an example of how to do it.

\subsubsection{controllers.py}
Controllers synthesize code from lower level game objects such as Characters, powers, and items,
in order to execute game logic. For example, Controllers
perform client-side linear interpolation to smooth out Character movement, this functionality is
intentionally separate from Characters because we want game objects to be as closely tied to the
game state as possible, and with as little auxiliary data as possible.

\codeword{PlayerController} is the controller for the player character. It should only be initialized
once during the lifetime of the player character. It drives client-side physics and sends updates
to the network, as well as client-side power usage. It also handles the camera, but this is
probably not good design. This will get separated out eventually.

\codeword{PlayerController} implements client-side prediction for player movement
to ensure smooth play while also enforcing server-authoritative position/rotation updates.
Unfortunately, the simulation is nondeterministic, so we need to adjust for the differences in the state.
The process for client-side prediction is as follows:\\
On every physics tick, we send the movement inputs from this tick to
the server, and then simulate the physics on the client. We store the
current position/rotation, and the ``targets'', and linearly interpolate
between them over the duration of the physics tick. This introduces a
physics tick worth of lag, but it's not noticeable and inconsequential
to the gameplay. This constitutes the ``prediction'' side of this story.

To accommodate the server updates, when we send the movement inputs, we
also include a sequence number that increments by one for each network update.
We map the sequence number to the current target state, and
when the server sends back the state, it also sends back the current
highest sequence number it has. Of course, when the state arrives back
to the client, it's in the past, so we look back in time to what our
target state was at the retransmitted sequence number. We compute the
difference, and add it to the current target. Crucially, we add the
displacement to the \codeword{Character}'s \codeword{displacement_components}
to be processed for collision logic. Without applying the collision logic, it's easy to clip through
the world and have an irrecoverable state.

\codeword{NPCController} is the controller for any client-side \codeword{Character} besides the
player character. It drives client-side movement/rotation interpolation for NPCs. This
interpolation is much more basic than the one implemented for player movement. Essentially,
whenever we receive a new state, we introduce some lag and interpolate towards it. 
\subsubsection{world.py}
This class should be used for creating all game objects that need to be sent over the
network. It provides methods for creating game objects from network data and augmenting
them with network-relevant information.
\subsubsection{UI Subpackage}
The UI provides a high-level interface into the game state. It has access to pretty much anything that
can affect the world. One potentially common pattern is that the game logic does something that
requires updating the UI code. Since the UI code is higher level than most of the game logic, it can't
access the UI code. So instead, the UI should provide callback functions to be executed, which will
update the UI. This is seen in action\_bar.py, where the function which creates the \codeword{Timer}
objects is set as a callback as \codeword{Power.on_use}.
\subsubsection{world\_responses.py}
This file defines all RPC's that are meant to be called by the server and executed on the client's
machine. Perhaps counterintuitively, the server does not need to have access to these functions,
at all. But the client does, because it's the one running them. This is the highest-level game logic
file, it has access to everything.
\subsubsection{input\_handler.py}
The other highest level game logic file is the input handler, which parses player inputs (aside from
the UI-level inputs) and performs game operations from those inputs. It is typically preferred
to act through the controllers where possible, but this is not necessary.
\subsection{Server Subpackage}
\subsubsection{character.py}
This file defines the server-side \codeword{ServerCharacter} class, which is a pretty small
deviation from the base \codeword{Character} class. See the client's character.py section
to see how ServerCharacters for players should be initialized.

Creating NPCs is similar, except the client doesn't need to be in the loop at the start, instead
the server just reads the data from the disk.

\subsubsection{effect.py}
This file defines \codeword{Effects}, which are the result of \codeword{Powers}.
There are several different kinds of effects - persistent/instant, harmful/beneficial.
Currently, Effect relies on the networking code, but this is subject to change soon.
\subsubsection{controllers.py}
This file defines the \codeword{MobController} class, which applies to all server-side
Characters (even player characters). It drives the server-side game logic for these
Characters, including physics, combat, and powers. Networking updates are sent to clients.
from here.
\subsubsection{world.py}
This class should be used for creating all game objects that need to be sent over the
network. It provides methods for creating game objects from network data and augmenting
them with network-relevant information.
\subsubsection{world\_requests.py}
This is the top-level game logic file for the server. It defines all RPCs called by
the client and executed on the server. As with world\_responses.py, the client does not
need to access this file at all.
\subsection{Project Root}
These are the \codeword{main.py}, \codeword{server.py}, \codeword{main_multiplayer.py}
files which pretty much just import everything necessary initialize the game. 
There is not currently any build process. This is subject to change.

The recommended way to launch the client is by simply running \codeword{main.py}, which
launches a child \codeword{server.py} process and connects to it automatically.
If you'd like to connect to the same local server on multiple clients, you can then run
\codeword{main_multiplayer.py} and press "c" to connect.

Alternatively, you may run \codeword{server.py} as its own process, and then connect
to it from \codeword{main_multiplayer.py}.

Of course, this launch process is subject to change once this becomes more of a real game.

\section{Development Principles}
I am not an expert software engineer, but I have picked up on good and bad patterns while developing
this game.
\begin{itemize}
    \item Don't try to account for things you're not certain you need to account for. Just like premature
        optimization is dangerous, premature generalization is just as evil. This is not to say that
        you should never think about the future when writing code, but there is a cutoff where trying
        to generalize new code does not give enough benefit to outweigh the costs (namely, spending
        extra time both upfront and when you need to refactor because you didn't generalize correctly,
        and also mental confusion). This is most prevalent when implementing the database side of
        features, you will never know every single intricacy of the feature you're just starting to
        implement, so don't waste too much time trying to account for problems that don't yet exist.
        Just do the dumb and simple solution at first, until you actually need to generalize.
    \item It's worthwhile to spend a few extra minutes thinking about the right way to do a big change,
        rather than jump right into it hoping for the best. Even more extreme, spending an hour
        solving a difficult code architecture problem is much better than spending 10 minutes figuring
        out how to do it dirty, but also having to spend a week refactoring it towards the good solution
        once the problems with the dirty solution arise.
    \item Thinking about other games is useful for inspiration, and coming up with ideas for features,
        but other games should never be the target. This leads to a lack of creativity in the solution
        to a problem, and is often lazy.
\end{itemize}


\section{Ursina Usage}
\subsection{Common Bugs}
\begin{itemize}
    \item ``Remote attempted to call a RPC that does not exist (maybe a name typo in the source code?)''\\
        This essentially just means that an unregistered RPC was called. Unfortunately it does not give
        the name of the procedure. Usually this is fixed by importing the relevant file that contains the
        RPC definition, fixing a typo, or implementing a non-existing RPC.
\end{itemize}
\end{document}
