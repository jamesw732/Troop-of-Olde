\documentclass{article}
\usepackage{graphicx}
\usepackage{enumerate}
\usepackage{hyperref}

\title{Troop of Olde Documentation}
\author{James Westbrook}

\begin{document}
\maketitle

\tableofcontents

\pagebreak

\section{Development Plan}
I expect pre-release development to basically take place in three stages.
I don't expect to confine
myself to these stages, but I think they will help structure my development. Currently
in the first stage, maybe like 60\% of the way through it.

First is the prototyping stage, where I simply try to implement the relevant features
of the game. Building combat mechanics, physics, networking, all that sort of stuff.
The key is functionality, not aesthetics. Animations and models will play a role in this stage,
but only in basic form. All art will be placeholder or lacking. UI is solely meant to provide an
interface between the user and functionality, and does not need to be pretty.
Checklist for this phase:
\begin{itemize}
    \item Networking
    \item Physics
    \item Auto attacking
    \item Powers (one of each type)
    \item Items
    \item Deaths/looting
    \item Basic stats functionality
    \item Basic skills functionality
    \item UI that interfaces all of the above
    \item Character model prototype and movement/attack animations
\end{itemize}
At the end of this phase, this will not be a game, it will be more like a blueprint for a game.
Should also consider closing the source code after declaring this phase to be done.

Second is the content generation stage. The basic idea here is to get a minimal working
example for the game. Once the basic prototypes and mechanics are nailed down,
it's time to start extending them into actual content used to build the world.
A necessary transition step here will be reconsidering the database. I don't think
it will be good to just use json files for everything in the final product, so the first
step before creating content will be to improve this. Since the plan is to make use of procedural
generation to make content, there will need to be a lot of time spent making this work during
this stage. Modeling, animations, and art will be a big part of this stage, but it will be
important to balance creating art with generating content. UI will probably not be a big
part of this phase, since the UI is mostly attached to the functionality.
Checklist for this phase:
\begin{itemize}
    \item Reconsider/improve database
    \item Create or generate world
    \item Create or generate items
    \item Create or generate NPCs
    \item Create powers
    \item Create launch screen for client and server
    \item Implement save system
    \item Add more models and character animations
    \item Add animations for powers (mostly just particles)
    \item Explore art style and maybe experiment with procedural art
    \item Add real art to the UI
\end{itemize}
At the end of this stage, there should be a playable game.
I should place a high standard on the quality of art entering the game. At the end of this stage,
it should look like a marketable product. It doesn't need to be a finished product, but I should
be able to release a Steam page and be able to find playtesters.

Third is the release stage. The goal here is to enter alpha and push out a finished product.
This is far-out still, so I'm guaranteed to miss a lot here, but I'm expecting some things to
be certain.
Checklist for this phase:
\begin{itemize}
    \item Create Steam page and market the game
    \item Improve quality of assets as needed
    \item Improve game balance
    \item Consider outsourcing cloud bandwidth to allow players to host servers remotely
\end{itemize}

\section{Code Infrastructure}
\subsection{Servers}
\subsubsection{Game Server}

\section{Ursina Usage}
\subsection{Common Bugs}
\begin{itemize}
    \item ``Remote attempted to call a RPC that does not exist (maybe a name typo in the source code?)''\\
        This essentially just means that an unregistered RPC was called. Unfortunately it does not give
        the name of the procedure. Usually this is fixed by importing the relevant file that contains the
        RPC definition, fixing a typo, or implementing a non-existing RPC.
\end{itemize}
\end{document}
