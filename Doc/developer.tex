\documentclass{article}
\usepackage{graphicx}
\usepackage{enumerate}
\usepackage{hyperref}
\usepackage{xparse}
\usepackage{amsmath}

\NewDocumentCommand{\codeword}{v}{%
\texttt{{#1}}%
}

\title{Troop of Olde Documentation}
\author{James Westbrook}

\begin{document}
\maketitle

\tableofcontents

\pagebreak

\section{Development Plan}
I expect pre-release development to basically take place in three stages.
I don't expect to confine
myself to these stages, but I think they will help structure my development. Currently
in the first stage, maybe like 60\% of the way through it.

First is the prototyping stage, where I simply try to implement the relevant features
of the game. Building combat mechanics, physics, networking, all that sort of stuff.
The key is functionality, not aesthetics. Animations and models will play a role in this stage,
but only in basic form. All art will be placeholder or lacking. UI is solely meant to provide an
interface between the user and functionality, and does not need to be pretty.
Checklist for this phase:
\begin{itemize}
    \item Networking
    \item Physics
    \item Auto attacking
    \item Sample Powers (one of each type)
    \item Items
    \item Robust NPC/player deaths
    \item Looting
    \item Learning powers from scrolls (and moving them on action bar)
    \item Basic stats functionality
    \item Basic skills functionality
    \item UI that interfaces all of the above
    \item Character model prototype and movement/attack animations
    \item Test connections on more than one device
    \item NPC AI
    \item Research necessary next steps such as art styles, blender,
    music composition, procedural generation
\end{itemize}
At the end of this phase, this will not be a game, it will be more like a blueprint for a game.
Should also consider closing the source code after declaring this phase to be done.

Second is the content generation stage. The basic idea here is to get a minimal working
example for the game. Once the basic prototypes and mechanics are nailed down,
it's time to start extending them into actual content used to build the world.
A necessary transition step here will be reconsidering the database. I don't think
it will be good to just use json files for everything in the final product, so the first
step before creating content will be to improve this. Since the plan is to make use of procedural
generation to make content, there will need to be a lot of time spent making this work during
this stage. Modeling, animations, and art will be a big part of this stage, but it will be
important to balance creating art with generating content. UI will probably not be a big
part of this phase, since the UI is mostly attached to the functionality.
Checklist for this phase:
\begin{itemize}
    \item Reconsider/improve database
    \item Create or generate world
    \item Create or generate items
    \item Create or generate NPCs
    \item Create powers
    \item Make physics more robust to the world, and consider improving the feel of movement.
    \item Create launch screen for client and server
    \item Implement save system
    \item Add more models and character animations
    \item Add animations for powers (mostly just particles)
    \item Explore art style and maybe experiment with procedural art
    \item Add real art to the UI
    \item Start composing music and making sound effects
    \item Improve network security/reliability
\end{itemize}
At the end of this stage, there should be a playable but incomplete game.
I should place a high standard on the quality of art entering the game. At the end of this stage,
it should look like a marketable product. It doesn't need to be a finished product, but I should
be able to release a Steam page and be able to find playtesters.

Third is the release stage. The goal here is to enter alpha and push out a finished product.
This is far-out still, so I'm guaranteed to miss a lot here, but I'm expecting some things to
be certain.
Checklist for this phase:
\begin{itemize}
    \item Create Steam page and market the game
    \item Improve quality of assets as needed
    \item Improve game balance
    \item Consider outsourcing cloud bandwidth to allow players to host servers remotely
    \item Finish composing music
\end{itemize}

\section{Code Infrastructure}
This section describes the key development choices made in the source code
for this game. This document is not intended to be a complete design
document, merely an aid for the developer, so I won't talk about everything
in here, but will try to hit the most important pieces. Code that seems
self-explanatory to me is ignored.
\subsection{Networking}
This game runs on a client-server framework. Currently, it is only tested to
support one device launching multiple clients.
All networking code is built on Ursina's
\codeword{networking} submodule, which is built on TCP and based on remote
procedure calls (RPCs).

All networking code is part of the \codeword{networking} submodule of this
repository. The structure of this code is under consideration, but below is a
brief description of the files in this directory:
\begin{itemize}
    \item \codeword{connect.py} is a small file that handles the work required
        when establishing a new connection to the server.
    \item \codeword{disconnect.py}, similarly, handles the work required when
        leaving the server. Disconnection logic is currently buggy, but this
        is a low priority at the moment.
    \item \codeword{network.py} defines the \codeword{Network} class, which is
        mostly just a centralized location for crucial network information
        such as the \codeword{Peer} instance and dicts mapping ids and connections
        to objects in memory. The \codeword{Network} object is added as a
        variable of the \codeword{GameState} object, so that RPCs can be accessed
        without having to import the corresponding files. This provides a great
        deal of separation between the internal game logic and the networking
        code at the cost of a global variable.
    \item \codeword{register.py} is a simple procedural file that registers all
        custom data types to be sent across the network.
    \item \codeword{world_requests.py} is a file containing all in-game networking
        requests sent by clients. All functions in this file are remote procedure
        calls that are called by clients, executed on the server. Relies on the
        game code and on \codeword{world_responses.py}.
    \item \codeword{world_responses.py} is a file containing all in-game networking
        responses sent by the server. All functions in this file are remote procedure
        item calls that are called by the server, executed on a client. Relies on the
        item game code.
\end{itemize}
To see how connections are made, it is best to look at some files in the root of the
repository.

The file \codeword{server.py} launches a headless Ursina app, and initializes
\codeword{network.peer} with \codeword{is_host=True} and a pre-set hsot name and port.
If run on its own, the user can launch \codeword{main_multiplayer.py} from a separate
terminal and press the "c" key to connect to the server.

Alternatively, one may simply launch \codeword{main.py}, which launches a child
\codeword{server.py} process and connects to it automatically.

\subsection{States}
States are essentially simplified versions of objects that are sendable over
the network. The typical usage is like:\\
Object $\rightarrow$ State $\rightarrow$ RPC call $\rightarrow$ State
$\rightarrow$ Object,\\
where the last arrow represents either an overwrite of a portion of the object's state,
or applying the state as a difference.

States are defined in \codeword{state_defs.py}. They are currently based on dicts,
but I'm realizing as I type this that this is unnecessary, and could be based on
lists. State definitions are basically mappings of variables to types, and as long as
the state is filled and the ordering is guaranteed and consistent between the
client and server, we can represent a State with just a list. Would need to convert
dicts to lists of tuples.
\subsection{Physics}
Implementing physics has proved to be challenging, but the current implementation seems
good.

There are currently three (well, 4, but I'll get to the fourth later) movement
factors. Player keyboard (WASD) movement, jumping, and gravity. Velocity
components are multiplied by the physics timestep (1/20th of a second) and then
added together, along with the displacement factors. After we obtain a single
velocity vector, we send it to the collision logic.

Collision rules are different when the character is grounded or not.
When not grounded, the character essentially just looks in the direction it's
falling, and if it would clip into the floor, it adjusts its movement to not clip
into the floor and sets the \codeword{grounded} boolean to True.
When grounded, it's a lot more complicated. We look ahead of where our initial
displacement would take us, and if there is anything in the way, first check
whether it's a wall or not (normalized $x$ component less than 0.2). If it's a wall,
we essentially move along the wall without changing our $y$ height.
The formula for this is obtained by a plane intersection, the intersection of the
plane $ax + by + cz = 0$ and $y = 0$. This yields the parameterization
\[
\begin{pmatrix}
    x\\
    y\\
    z 
\end{pmatrix} =
\begin{pmatrix}
    b\\0\\-a
\end{pmatrix}s,\]
and rescaling this and multiplying by the dot product gives us the displacement
we want.

When the colliding surface is a feasible slope to walk up, we instead keep
the $(x, z)$ direction from our original movement, and deduce the correct
$y$ from this and the plane normal. This is the same as finding the
intersection of the planes $ax + by + cz = 0$ and the plane defined by
$(0, 0, 0), (0, 1, 0), (\alpha, 0, \beta)$ (we look straight up because gravity goes
straight down), where $(\alpha, \beta)$ is the $(x, z)$ direction of our
original displacement. Then our line is defined by
\[
\begin{pmatrix}
    x\\y\\z
\end{pmatrix} = 
\begin{pmatrix}
    \alpha b\\-\beta c - \alpha a\\\beta b
\end{pmatrix}s.\]
Rescaling to match our original displacement gives us our final result.
\subsection{Characters and Controllers}
All characters in the game - NPCs, player characters, and on the client vs server,
are all represented by the same Character class. The Character class should actually
be more or less the intersection of all these functionalities. The differences
are captured by their controllers.

The PlayerController is responsible for handling most client-side inputs
and sending them to the server. This class is not the complete authority
for this, as some inputs are handled by the UI code, but pretty much any
generic input is handled here. Handles client-side prediction for character
movement (explained below).

The NPCController actually doesn't do much controlling, it's what controls
every character besides the player character. NPCs and other players are handled
pretty much the same. The most complicated functionality of this class is
linearly interpolating between states received by the server.

The MobController class is the server-side controller class. It's responsible
for handling all server-side character computations, such as movement and
combat. Relays combat state changes and movement updates to the clients.
\subsubsection{Client-Side Prediction}
To ensure smooth play while enforcing server-authoritative character movement,
we simulate the game physics client-side. Unfortunately, the simulation
is nondeterministic, so we need to adjust for the differences in the state.
The process for client-side prediction is as follows:\\
On every physics tick, we send the movement inputs from this tick to
the server, and then simulate the physics on the client. We store the
current position/rotation, and the ``targets'', and linearly interpolate
between them over the duration of the physics tick. This introduces a
physics tick worth of lag, but it's not noticeable and inconsequential
to the gameplay. This constitutes the ``prediction'' side of this story.

To accommodate the server updates, when we send the movement inputs, we
also include a sequence number that increments by one each physics
tick. We map the sequence number to the current target state, and
when the server sends back the state, it also sends back the current
highest sequence number it has. Of course, when the state arrives back
to the client, it's in the past, so we look back in time to what our
target state was at the retransmitted sequence number. We compute the
difference, and add it to the current target. Crucially, we add the
displacement to the \codeword{Character}'s \codeword{displacement_components}
to be processed for collision logic. Without applying the collision logic, it's easy to clip through the world and have an irrecoverable state.


\section{Ursina Usage}
\subsection{Common Bugs}
\begin{itemize}
    \item ``Remote attempted to call a RPC that does not exist (maybe a name typo in the source code?)''\\
        This essentially just means that an unregistered RPC was called. Unfortunately it does not give
        the name of the procedure. Usually this is fixed by importing the relevant file that contains the
        RPC definition, fixing a typo, or implementing a non-existing RPC.
\end{itemize}
\end{document}
